%<*profile>
\section{\texorpdfstring{\faUser~Profile}{~Profile}}

  \begin{educationEntry} Passionate about coding, machine learning, and statistics. Completed my PhD \href{\detokenize{https://atrium.lib.uoguelph.ca/server/api/core/bitstreams/8ed1891d-58ca-457b-b0d4-33a015fb2db9/content}}{\textcolor{blue}{dissertation}} at the intersection of these fields to address a practical problem in Industrial-Organizational psychology and received the 2022/2023 Canadian Psychological Association Certificate of Academic Excellence for this work. In my dissertation, I coded and ran Monte Carlo simulations on an AWS instance to evaluate the performance of nonlinear longitudinal models. Writes white papers on machine learning topics at \href{https://sebastiansciarra.com}{\textcolor{blue}{sebastiansciarra.com}}. \newline\end{educationEntry}

  
%</profile>
%<*skills>
\section{\texorpdfstring{\faCogs~Skills}{~Skills}}

  \fontsize{8}{8} \selectfont

  \begin{tabularx}{\textwidth}{@{} p{0.25\rightcolumnwidth} p{0.2\rightcolumnwidth} p{0.72\rightcolumnwidth} @{}}
                         \textbf{Coding languages}
                         \begin{itemize}[noitemsep,topsep=0pt,partopsep=0pt,parsep=0pt,leftmargin=9pt,label=\raisebox{0.04cm}{\fontsize{3pt}{3pt}$\blacksquare$}]
                         \setstretch{1.25} \item Python\item R\item SQL\item \LaTeX\item Javascript\item HTML\item CSS\end{itemize}
                         %second column
                         &
                         \textbf{IDEs/platforms}
                         \begin{itemize}[noitemsep,topsep=0pt,partopsep=0pt,parsep=0pt,leftmargin=9pt,label=\raisebox{0.04cm}{\fontsize{3pt}{3pt}$\blacksquare$}]\item PyCharm\item RStudio\item AWS\item Git\item MySQL\end{itemize}
                        %third column
                        &
                        \textbf{Technical skills}
                        \begin{itemize}[noitemsep,topsep=0pt,partopsep=0pt,parsep=0pt,leftmargin=9pt,label=\raisebox{0.04cm}{\fontsize{3pt}{3pt}$\blacksquare$}]\item Data visualization (ggplot2, plotnine)\item Data cleaning (tidyverse, pandas, numpy)\item Machine learning (e.g., regularized regression, decision trees, random forests, mixture models)\item Statistics (e.g., latent variable models, factor analysis, multilevel modelling)\end{itemize}
                        \end{tabularx}

  
%</skills>
%<*education>
\section{\texorpdfstring{\faUniversity~Education}{~Education}}

  \begin{educationEntry}   \fontsize{8}{8}\selectfont \textbf{PhD \textbar{} Industrial-Organizational Psychology} \newline University of Guelph \newline Sep. 2018--May 2023 \newline \end{educationEntry}\begin{educationEntry}   \fontsize{8}{8}\selectfont \textbf{MSc \textbar{} Cognitive Psychology} \newline McMaster University \newline Sep. 2016--June 2018 \newline \end{educationEntry}\begin{educationEntry}   \fontsize{8}{8}\selectfont \textbf{Honours BSc \textbar{} Psychology, Neuroscience \& Behaviour} \newline McMaster University \newline Sep. 2012--June 2016 \newline \end{educationEntry}

  
%</education>
%<*selected-white-papers>
\section{\texorpdfstring{\faCoffee~Selected white
  papers}{~Selected white papers}}

  \fontsize{8}{8}\selectfont \href{https://sebastiansciarra.com/technical_content/understanding_ml/}{\textbf{The Game of Supervised Machine Learning: Understanding the Setup, Players, and Rules} \newline {\fontsize{6.5}{6.5}\selectfont Published \newline \textit{10 August 2023}}}
  \newline \newline \fontsize{8}{8}\selectfont \href{https://sebastiansciarra.com/technical_content/em/}{\textbf{The Expectation-Maximization Algorithm: A Method for Modelling Mixtures of Distributions} \newline {\fontsize{6.5}{6.5}\selectfont Published \newline \textit{28 April 2023}}}
  \newline \newline \fontsize{8}{8}\selectfont \href{https://sebastiansciarra.com/technical_content/mle/}{\textbf{Probability, Likelihood, and Maximum Likelihood Estimation} \newline {\fontsize{6.5}{6.5}\selectfont Published \newline \textit{19 March 2023}}}
  \newline \newline

  
%</selected-white-papers>
%<*employment-experience>
\section{\texorpdfstring{\faSuitcase~Employment
  ~experience}{~Employment ~experience}}

  \begin{resumeEntry}{Teaching Assistant}{University of Guelph}{Sep. 2018--May 2023}[Created R scripts for assignments and taught labs for the following courses in measurement and statistics: \begin{itemize}   \fontsize{8}{8}\selectfont
                  \item PSYC 3290 (Conducting Statistical Analyses in Psychology)
                  \item PSYC 3250 (Psychological Measurement) 
                  \item PSYC 6060 (Research Design and Statistics)
                  \item PSYC 6380 (Psychological Applications of Multivariate Analysis)
              \end{itemize}][Taught a variety of topics in methdods and statistics (e.g., regression with continuous and categorical [i.e., ANOVA] variables, \textit{p} values, \textit{p} hacking, hierarchical linear modelling, factor analysis, latent variable modelling, etc.)][][][]\end{resumeEntry}

  \begin{resumeEntry}{Graduate Research Assistant}{University of Guelph (Part-Time)}{Sep. 2020--Apr. 2021}[Used R to clean data, compute descriptive statistics, and run regression analyses (with categorical and/or continuous variables) for organizational data on turnover, downsizing, and growth][][][][]\end{resumeEntry}

  \begin{resumeEntry}{Consultant}{Geosyntec (Part-Time)}{Sep. 2020--Dec. 2020}[Worked with a team of graduate students to improve the interview procedure][Developed customized recommendations to structure the interview procedure so that adverse hiring outcomes were reduced and skills were more rigorously evaluated][][][]\end{resumeEntry}

  \begin{resumeEntry}{Consultant}{Schema App (Part-Time)}{Jan. 2020--Apr. 2020}[Worked with a team of graduate students to improve the onboarding of new employees][Synthesized customized recommendations by using literatures on realistic job previews, goal setting, and mentoring][][][]\end{resumeEntry}

  
%</employment-experience>
%<*data-science-experience>
\section{\texorpdfstring{\faCode~Data science
  ~experience}{~Data science ~experience}}

  \begin{resumeEntry}{\href{https://github.com/sebsciarra/smltheory}{smltheory}}{Python package}{Aug. 2023}[Package contains nine modules and 30 functions][Functions within package simulate data sets and demonstrate propositions of supervised machine
                   learning propositions (e.g., bias-variance tradeoff, excess risk decomposition)][][][]\end{resumeEntry}

  \begin{resumeEntry}{\href{https://github.com/sebsciarra/cobaltResume}{cobaltResume}}{R Package}{May 2023}[Automates generation of resumes and cover letters within RStudio][A template and class file were created (\textapprox 700 lines of \LaTeX code) to specify a styling template that draws inspiration from the cobalt theme in the RStudio IDE][R functions were created to easily generate resume entries and merge resume and cover letters into one PDF file][][]\end{resumeEntry}

  \begin{resumeEntry}{\href{https://sebastiansciarra.com}{sebastiansciarra.com}}{Personal website}{Mar. 2023}[Used HTML, JavaScript, and CSS to create a personal website for writing white papers][White papers focus on statistics, machine learning, and coding by explaining technical details, providing demonstrations, and conducting simulation experiments][White papers use code from a variety of languages to explain content. As an example, my post titled ``\href{https://sebastiansciarra.com/coding_tricks/em_demo/}{\textcolor{blue}{Coding and Visualizing the Expectation-Maximization Algorithm}}" used R, Python, and CSS code][][]\end{resumeEntry}

  \begin{resumeEntry}{\href{https://github.com/sebsciarra/guelphdown}{guelphdown}}{R Package}{Mar. 2023}[Created an R package that automates the generation of theses according to the University of Guelph formatting requirements][A template and class file were created (\textapprox 1400 lines of \LaTeX code) to specify formattings for the preamble, body, references, and appendices][An example of the formatting can be seen in my  \href{\detokenize{https://atrium.lib.uoguelph.ca/server/api/core/bitstreams/8ed1891d-58ca-457b-b0d4-33a015fb2db9/content}}{\textcolor{blue}{thesis}}][][]\end{resumeEntry}

  \begin{resumeEntry}{\href{https://github.com/sebsciarra/nonlinSimsAnalysis}{nonlinSimsAnalysis}}{R Package}{Mar. 2022}[Package contains 105 functions][Functions automate the cleaning, analysis, and visualization of large data sets (e.g., 40 000+ rows) for my doctoral dissertation][The creation of several different types of tables and figures were automated by this package][][]\end{resumeEntry}

  \begin{resumeEntry}{\href{https://github.com/sebsciarra/nonlinSims}{nonlinSims}}{R Package}{Jan. 2022}[Package contains 30 functions][Functions run the simulation experiments of my doctoral dissertation][The peformance of nonlinear longitudinal models are evaluated (e.g., structured latent growth curve models) are evaluated under several conditions][][]\end{resumeEntry}

  \begin{resumeEntry}{\href{https://github.com/sebsciarra/learning_SQL}{Learning SQL}}{Project}{Mar. 2021}[Went through 16 of 18 chapters from Alan Beaulieu's \href{https://www.amazon.ca/Learning-SQL-Generate-Manipulate-Retrieve/dp/1492057614/ref=sr_1_1?hvadid=324956203165&hvdev=c&hvlocphy=9000835&hvnetw=g&hvqmt=e&hvrand=9921433988929165270&hvtargid=kwd-312865785332&hydadcr=16084_10268182&keywords=learning+sql+by+alan+beaulieu&qid=1684158363&sr=8-1}{\textcolor{blue}{Learning SQL}}][Topics include filtering, querying multiple tables, sets, grouping and aggregates, subqueries, joins, transactions etc.][][][]\end{resumeEntry}



%</data-science-experience>
